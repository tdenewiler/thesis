% No symbols, formulas, superscripts, or Greek letters are allowed
% in your title.
\title{Robot Traffic School}
\author{Thomas Denewiler}
\degreeyear{2011}
\degree{Master of Science}
\field{Mechanical and Aerospace Engineering}
\chair{Professor Thomas R. Bewley}

% Uncomment the next line iff you have a Co-Chair
% \cochair{Professor Cochair Semimaster} 
%
% Or, uncomment the next line iff you have two equal Co-Chairs.
%\cochairs{Professor Chair Masterish}{Professor Chair Masterish}

%  The rest of the committee members  must be alphabetized by last name.
\othermembers{
Professor Raymond de Callafon \\ 
Professor Ryan Kastner \\
}
\numberofmembers{3} % |chair| + |cochair| + |othermembers|

%% START THE FRONTMATTER
\makeatletter
\let\@currsize\normalsize
\begin{frontmatter}
\makefrontmatter

%% DEDICATION
% You have three choices here:
%   1. Use the ``dedication'' environment.   Put in the text you want,
%   and you'll get a perfectly respectable dedication page.
%
%   2. Use the ``mydedication'' environment.  If you don't like the
%   formatting of option 1, use this environment and format things
%   however you wish.
%
%   3. If you don't want a dedication, it's not required.
\begin{dedication} % The style file will format this for you.
To Grandma Denny, \\
it's a small step from board games to Kalman filters, \\
To my parents, \\
for their time and encouragement in everything.
\end{dedication}

% \begin{mydedication} % You are responsible for formatting here.
%   \vspace{1in}
%   \begin{flushleft}
% 	To me.
%   \end{flushleft}
%
%   \vspace{2in}
%   \begin{center}
% 	And you.
%   \end{center}
%
%   \vspace{2in}
%   \begin{flushright}
% 	Which equals us.
%   \end{flushright}
% \end{mydedication}


%% EPIGRAPH
%  The same choices that applied to the dedication apply here.

% \begin{epigraph} % The style file will position the text for you.
%   \emph{A careful quotation\\
%   conveys brilliance.}\\
%   ---Smarty Pants
% \end{epigraph}

% \begin{myepigraph} % You position the text yourself.
%   \vfil
%   \begin{center}
%     {\bf Think! It ain't illegal yet.}
%
% 	\emph{---George Clinton}
%   \end{center}
% \end{myepigraph}

\tableofcontents
\listoffigures  % Uncomment if you have any figures
\listoftables   % Uncomment if you have any tables


%% ACKNOWLEDGEMENTS
%  While technically optional, you probably have someone to thank.
%  Also, a paragraph acknowledging all coauthors and publishers (if
%  you have any) is required in the acknowledgements page and as the
%  last paragraph of text at the end of each respective chapter. See
%  the OGS Formatting Manual for more information.

\begin{acknowledgements}
The enthusiasm and energy of Professor Thomas Bewley has been an inspiration and his support and advice have been invaluable.

I am grateful to Gideon Prior, Nima Ghods, Amin Rahimi and Steve Stancliff for our many long conversations while learning how to build better robots.

I would also like to thank Mike Bruch, Bart Everett and the ACS team (Gaurav Ahuja, Donnie Fellars, Greg Kogut and Brandon Sights) as well as Jason Lum, Kelly Grant and the EOD technicians at SPAWAR for their support with both hardware and software.

And without the love and encouragement from Silvie Georgens I would not have made it this far.
\end{acknowledgements}


%% VITA
%  A brief vita is required in a doctoral thesis. See the OGS
%  Formatting Manual for more information.
% \begin{vitapage}
% \begin{vita}
%   \item[2002] B.~S. in Mathematics \emph{cum laude}, University of Southern North Dakota, Hoople
%   \item[2002-2007] Graduate Teaching Assistant, University of California, San Diego
%   \item[2007] Ph.~D. in Mathematics, University of California, San Diego
% \end{vita}
% \begin{publications}
%   \item Your Name, ``A Simple Proof Of The Riemann Hypothesis'', \emph{Annals of Math}, 314, 2007.
%   \item Your Name, Euclid, ``There Are Lots Of Prime Numbers'', \emph{Journal of Primes}, 1, 300
% 	B.C.
% \end{publications}
% \end{vitapage}

%% Abstract
% Doctoral dissertation abstracts should not exceed 350 words. MS thesis
% abstracts can be up to 250 words. The abstract may, however, continue to a
% second page if necessary.
\begin{abstract}
Increasing the autonomous performance of robots increases the complexity of possible behaviors that can be implemented and the utility of the robots to end users. To that end the state estimation and control algorithms for EOD differential drive robots have been studied and improved by generating better noise models for the extended Kalman filter using a discriminative training method and implementing a nonlinear model based controller. A DGPS system was temporarily integrated with the standard robot sensors to measure ground truth positions which were used to find more accurate system and measurement noise covariance values for use with the Kalman filter. Independently, a control Lyapunov function was found based on differential drive robot kinematics to output a suitable control law and the results are compared with the previously existing PID controller, especially in regards to the benefits of navigation at varying speeds. Several practical considerations of tuning the model based controller are explored and recommendations are given for several different operating scenarios. The improved robot performance was implemented using several different robots and software that are currently fielded and will accelerate development of advanced maneuvers such as retrotraverse over long distances and obstacle avoidance.
\end{abstract}

\end{frontmatter}
\makeatother

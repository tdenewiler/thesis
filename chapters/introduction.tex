\chapter{Introduction}
\label{ch:introduction}
In this chapter we give a brief overview of autonomous navigation and the small Explosives Ordinance Disposal (EOD) robots that will be considered in this thesis. We also describe some of the issues in state estimation and controls that were considered for improving the autonomous driving behaviors of the robots.

\section{The Need for Autonomy}
\label{sec:needforautonomy}
Robots have been developed to assist humans in tasks that are generally considered dirty, dangerous or boring. Recently, robots have found a useful niche as a tool to help EOD teams assess and eliminate threats from improvised explosive devices (IEDs), commonly referred to as roadside bombs. The use of robots allows humans to maintain a safe stand-off distance while investigating a scene. Clearly, this falls under the dangerous category. The current method that EOD teams use involves teleoperation of the robot from the base of operations to the object of interest. The teleoperation task consumes the operators energy and focus while they are navigating the robot to and from a goal location. In the process of teleoperating a robot the operator is exposed and vulnerable to external threats including ambushes and enemy snipers.

As technologies mature they provide humans with better tools. However, as the use of robots increases shortcomings are discovered, such as the vulnerability due to teleoperation, and that opens up avenues for improvements. One approach to reducing the amount of work humans are required to perform is to give the robots more intelligence via autonomous behaviors. This is accomplished using additional sensors, specialized actuators and more advanced software to automate as many routine tasks as possible. In time, more complex tasks such as navigation over rough terrain, around obstacles and back to a home location will become routine and automated as well, thus freeing up the operators to focus on higher level tasks.

When adding autonomy to robots nearly all of the tasks can be summarized by the following questions:
\begin{itemize}
\item Where am I?
\item What's around me?
\item Where do I want to go?
\item How do I get there?
\end{itemize}

The initial attempt at adding autonomy to EOD robots resulted in somewhat erratic driving behavior. This was especially evident near obstacles as the robot trajectory would not be smooth while changing speeds when attempting to move around the obstacle \cite{Bruch00}. In this thesis we will look at smoothing out the trajectories taken by the robot by making improvements to the state estimation (Where am I?) and controls (How do I get there?) algorithms. This work ignores actual obstacle detection (What's around me?) and will be using a simple planning algorithm (Where do I want to go?) to simulate obstacles in the robot path which will force the robot to change direction and speed multiple times. One of the benefits of a new controls algorithm with respect to planning will be discussed as well.

Other tasks for small robots include sending them into buildings that are dangerous due to structural damage or unknown, possibly hostile elements inside. The goal is to have the robots map the interior and provide operators with images to assess the danger prior to any humans entering the buildings \cite{CongressUGV06}. After the attacks on the World Trade Center on September 11, 2001, several small robot systems were used to look for survivors in the rubble and to help assess structural damage to nearby buildings \cite{Everett02}.

For all of the tasks that robots are assigned speed, efficiency and precision are valuable characteristics that can result in the success or failure of a mission. Improved autonomous navigation would improve all of those characteristics in small robot systems. An example of an advanced behavior that becomes possible with better navigation is the ability to have a robot retrotraverse, meaning to find its way back to a preset location without human intervention. The benefits of improved navigation include less time waiting for a robot to get close to an IED and clearing a building. This would result in less time for humans in a hostile or dangerous environment. In search and rescue situations this would lead to less time for searching and more time for rescuing.

\section{Thesis Outline \& Contributions}
\label{sec:outline}
The problem of autonomous navigation is not isolated to any one technical area. Instead, it is a combination of estimating the robots position and environment as well as controlling the robots trajectory. As much as possible this research has attempted to isolate the effects of each of those areas to enable a quantitative analysis to determine which parts of the system contribute the largest effect on overall robot behavior.

Chapter \ref{ch:background} gives some background on the types of robots used, the sensors on the robots, the operator control unit (OCU) and the testing area. Chapter \ref{ch:estimation} presents the state estimation algorithm that is used on the robots and techniques for improving the state estimate. Chapter \ref{ch:controls} discusses the original control algorithm and a new control algorithm applied as part of this research. Chapter \ref{ch:results} gives the results of experiments ran with the robots and looks at the main contributing factors for smooth autonomous navigation. Chapter \ref{ch:futurework} suggests future avenues of research to pursue. The conclusion is found in Chapter \ref{ch:conclusion}.

The main contribution of this research is that the small robots investigated here have greater autonomy which allows for less human oversight of basic functionality. The improvements to autonomy are a result of the analysis of and fixing bugs in the state estimation algorithm as well as a new control algorithm. The new model based controller has been investigated, and guidelines are given for how to configure the design variables according to the operating environment of the robots. The algorithms developed during this reasearch have been implemented on fielded, production systems, and are shown to work better than the original algorithms. Additionally, these algorithms were shown to work on several different robots without additional tuning.

This work is especially relevant for the dirty, dangerous and boring tasks where robots are most useful because it allows humans to focus their concentration, time and effort on being clean, safe and efficient.

\chapter{Future Work}
\label{ch:futurework}

\begin{itemize}
\item Use the results from this work to make retrotraverse work better. One way this can be accomplished is that obstacles should be easier to avoid since the robot position is more accurate so following a previously navigated path after some time has passed should result in less drift so previously avoided static obstacles will be closer to where they were before. Additionally, a control law that works at varying speeds (like model based, unlike PID that had to be tuned with different gains for different speeds) will make navigation around obstacles smoother and faster.
\item Work on a planner that can automatically tune the gains for the model based controller based on the environment of the robot. For example, if there are no nearby obstacles then let the path curvature be large. If the environment is cluttered then force the path curvature to be much smaller. Similarly, a planner that can select the best controller out of PID, fuzzy logic and model based would likely improve the navigation characteristics of the robots.
\item Use the learning algorithm for indoor robots as there is no reason why a robot needs to have GPS to benefit from using the DGPS system for training the $Q$ and $R$ matrices in the EKF. A test area would need to be set up outdoors so that DGPS can be used and the normal sensors also work by bouncing off walls and what not but the DGPS ground truth position can be converted to the local coordinate system and used to generate an error metric for the EKF position output.
\item Use a high quality IMU and compass that serve as ground truth for Euler angles in addition to the DGPS system so that the training algorithm attempts to minimize the errors between Euler angles in addition to position.
\item Implement the more advanced work done by \cite{Lapierre06} and \cite{Gulati08} to improve the performance of the model based controller. In particular \cite{Gulati08} uses control Lyapunov functions that attempt to constrain the accelerations of the vehicle and uses splines to generate intermediate waypoints between the vehicles current position and the goal position.
\item Extend the model based controller to have constraints that allow the controller to work with non-unicycle like vehicles such as those that have Ackerman steering like most personal automobiles which are not physically able to rotate in place.
\end{itemize}

\chapter{Future Work}
\label{ch:futurework}

\begin{itemize}
\item Use the Kalman filter learning algorithm, from Section \ref{sec:kftrainingparams}, for indoor robots. The robots do not require a GPS sensor to benefit from the learning algorithm. If a suitable outdoor environment can be set up then DGPS can still be used as ground truth. A suitable environment could be a mock building without a roof. This would allow the DGPS receiver to get position measurements while simultaneously letting laser range finders have nearby objects to sense (assuming that the indoor robots are using lasers to perform some type of simultaneous localization and mapping algorithm (SLAM) that is common on current robots).
\item Use a high quality IMU and compass, that serve as ground truth for Euler angles, in addition to the DGPS system so that the training algorithm attempts to minimize the errors between Euler angles in addition to position.
\item Implement the more advanced work done by \cite{Lapierre06} and \cite{Gulati08} to improve the performance of the model-based controller. In particular, \cite{Gulati08} uses control Lyapunov functions that attempt to constrain the velocities and accelerations of the vehicle. They also use splines to generate intermediate waypoints between the vehicles current position and the goal position so that the route is smooth. Similarly, \cite{Burgard09} uses quintic splines to generate routes.
\item Extend the model-based controller to have constraints that allow the controller to work with non-differential drive vehicles such as those with Ackerman steering, like most personal automobiles \cite{Shiller91dynamicmotion}. These other types of vehicles are not physically able to rotate in place.
\item Build on the current extended Kalman filter in ACS to implement an unscented Kalman filter as described by \cite{ThrunProbRobots06} and \cite{Orderud05} to better handle the nonlinearities of the system.
\end{itemize}

\chapter{Future Work}
\label{ch:futurework}

\begin{enumerate}
\item Use the results from this work to make retrotraverse work better. One way this can be accomplished is that obstacles should be easier to avoid since the robot position is more accurate so following a previously navigated path after some time has passed should result in less drift so previously avoided static obstacles will be closer to where they were before. Additionally, a control law that works at varying speeds (like Lyapunov, unlike PID that had to be tuned with different gains for different speeds) will make navigation around obstacles smoother and faster.
\item Work on a planner that can automatically tune the gains for the Lyapunov controller based on the environment of the robot. For example, if there are no nearby obstacles then let the path curvature be large. If the environment is cluttered then force the path curvature to be much smaller.
\item Use the learning algorithm for indoor robots as there is no reason why a robot needs to have GPS to benefit from using the DGPS system for training the $Q$ and $R$ matrices in the EKF. A test area would need to be set up outdoors so that DGPS can be used and the normal sensors also work by bouncing off walls and what not but the DGPS ground truth position can be converted to the local coordinate system and used to generate an error metric for the EKF position output.
\item Add a high quality IMU that serves as ground truth for Euler angles to the DGPS system so that the training algorithm attempts to minimize the errors between Euler angles in addition to position.
\item Develop a better method for training the EKF covariance matrices by using a better parameter search algorithm or by massively parallelizing the current naive, brute force algorithm. Since determining all of the possible matrices is relatively inexpensive compared to simulating the EKF output all of the possible matrices for a particular grid size can be generated and stored in memory and then the EKF can be run simultaneously using each of those matrices where the next grid step uses the $Q$ and $R$ that minimized the output error of the EKF.
\item Characterize individual IMUs better since that is a large source of error.
\item Instead of the naive, brute force method of trying all the $Q$ and $R$ matrices use a better machine learning algorithm. This could either be a neural network using FANN or a gradient descent approach that leads the elements of $Q$ and $R$ in the proper direction when minimizing the derivative of the errors computed using either the residual or the prediction error metrics. It would be great to try all of the different methods, have theory showing which ones should work better and then have data to show how well the results match with the theory. Some analysis of which methods work best and under what circumstances would be hugely beneficial, not to mention interesting.
\item Try LQR/LQG controllers and compare them.
\item Log input and output data from the robots and apply the general realization algorithm \cite{deCallafonGRA08} to generate a state space model. Then use that model with LQG.
\item Mention that the parameter training of \cite{Abbeel-RSS-05} was recently used by \cite{SakaiKuroda10}.
\item Say I want to implement the more advanced work done by \cite{Lapierre06} and \cite{Gulati08} to improve the performance of the model-based controller.
\end{enumerate}

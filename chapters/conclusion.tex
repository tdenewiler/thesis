\chapter{Conclusion}
\label{ch:conclusion}
As shown in Chapter \ref{ch:results}, the estimation and controls algorithms developed and implemented as part of this thesis resulted in significant improvements to the existing algorithms used for EOD robots. The Kalman filter showed a significant improvement over initial results. The state estimate improvement was due to a combination of fixing errors in the implementation of the Kalman filter and training the covariance values of the noise models.

The model-based controller was shown to drive at multiple velocities, across different surfaces, with no gain tuning required. This is contrary to the original PID controller. Additionally, the model-based controller exhibits smooth deceleration as it approaches waypoints, which is a consequence of the control law and not due to external commands to slow down like the PID controller. These improvements to robot navigation have been implemented in ACS and are now running on fielded systems to the benefit of end users.

There now exist better answers to the questions of "Where am I?" and "How do I get there?" due to the results obtained in this thesis.

\section{Future Work}
\label{sec:futureWork}
During the course of the research performed for this thesis several areas were not investigated as fully as they could have been. Other methods are also available that could offer further improvements to the state estimation and controls algorithms. Finally, areas other than those discussed in this thesis could benefit from application of the results obtained here. These include:
\begin{itemize}
\item Use the Kalman filter learning algorithm, from Section \ref{sec:kftrainingparams}, for indoor robots. The robots do not require a GPS sensor to benefit from the learning algorithm. If a suitable outdoor environment can be set up then DGPS can still be used as ground truth. A suitable environment could be a mock building without a roof. This would allow the DGPS receiver to get position measurements while simultaneously letting laser range finders have nearby objects to sense (assuming that the indoor robots are using lasers to perform some type of simultaneous localization and mapping algorithm (SLAM) that is common on current robots).
\item Use a high quality IMU and compass, that serve as ground truth for Euler angles, in addition to the DGPS system so that the training algorithm attempts to minimize the errors between Euler angles in addition to position.
\item Implement the more advanced work done by \cite{Lapierre06} and \cite{Gulati08} to improve the performance of the model-based controller. In particular, \cite{Gulati08} uses control Lyapunov functions that attempt to constrain the velocities and accelerations of the vehicle. They also use splines to generate intermediate waypoints between the vehicles current position and the goal position so that the route is smooth. Similarly, \cite{Burgard09} uses quintic splines to generate routes.
\item Extend the model-based controller to have constraints that allow the controller to work with non-differential drive vehicles such as those with Ackerman steering, like most personal automobiles \cite{Shiller91dynamicmotion}. These other types of vehicles are not physically able to rotate in place.
\item Build on the current extended Kalman filter in ACS to implement an unscented Kalman filter as described by \cite{ThrunProbRobots06} and \cite{Orderud05} to better handle the nonlinearities of the system.
\item Begin using the known outputs of the controller, or teleoperation commands, as inputs to the Kalman filter. This will change the system model in the prediction update step (\ref{eq:kfpredictionupdate}) to
\begin{align*}
\hat{x}_{k+1}^- = \Phi_k \hat{x}_k^+ + G_ku_k
\end{align*}
where $G_k$ is the control-input model that maps the control outputs to state variables.
\end{itemize}

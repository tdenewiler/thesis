\chapter{Background}
\label{ch:background}
*** Put in a description of the PackBot, Talon and Urbot along with a description of the algorithms originally used and the results obtained using those algorithms. Talk about JAUS and MOCU a little bit. Discuss more about how the robots are used by EOD. Include that for robots to drive around on their own all they require is an estimate of where they are, a path to follow, a controller to determine the actuator outputs and motor controllers to perform the controller outputs. ***

*** It might be best to have more of a description of the actual problem here along with a description of the testing area. ***

\section{Small Unmanned Ground Vehicles}
The PackBot is manufactured by iRobot. *** Add more. ***

The Talon is manufactured by Foster Miller. *** Add more. ***

The Urbot was an experimental prototype created at SSC-SD. *** Add more. ***

\section{MOCU \& JAUS}
The Multi-Robot Operator Control Unit (MOCU) is a highly configurable front-end for simultaneous command and control of multiple systems and was created at SSC-SD \cite{PowellMOCU08}. MOCU has the ability to use a variety of communications protocols for interfacing to different systems and uses the Joint Architecture for Unmanned Systems (JAUS) to send and receive data to all of the UGVs used in this research \cite{RoweJAUS08}.

\section{The Duals: Estimation \& Controls}
It is very difficult to simply work on either state estimation or controls individually as there is a large amount of coupling between the two areas. Although the main goal is to make the robots drive more smoothly and that the actuator and motor outputs are ultimately generated by the control system it is still the case that the role of state estimation is equally important. If there exists large meaurement errors, drift or bias in the sensor readings then the robot will not have a very good idea of where it is locted and there will not be a controller that can stabilize the system. *** Talk about observability and controllability. Mention theory that shows link between estimation and control. ***

An example would be when the only sensor available for measurements is an IMU which suffers from drift and bias, where both effects are exagerrated by temperature. There have been situations in which an IMU was in a robot with the motors turned off so that the robot is not moving. However, due to excessive heat in the electronics bos the IMU measurements report that the heading of the robot keeps moving in circles at a rate of $\frac{\pi}{5} rad/s$. With a controller that was known to keep the robot stable when the IMU was working properly started forcing the robot to turn in circles when the motors were turned on even thought the command was to stay in one place. This shows the importance of state estimation on overall robot performance -- it is not enough to only have a good controller.
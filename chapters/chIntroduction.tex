\chapter{Introduction}
\label{ch:introduction}
In this chapter we give a brief overview of autonomous navigation and the small unmanned ground vehicles (UGVs) that will be considered in this thesis. We also describe some of the issues in state estimation and controls that were considered during research in improving the autonomous driving behaviors of the UGVs.

\section{The Need for Autonomy}
\label{sec:needforautonomy}
Robots have been developed to assist humans in tasks that are generally considered dirty, dangerous or boring. Recently robots have found a useful niche as a tool to help Expolosives Ordinance Disposal (EOD) teams to assess and eliminate threats due to improvised explosive devices (IEDs), commonly referred to as roadside bombs, by allowing humans to maintain a safe stand-off distance while investigating a scene. Clearly this falls under the dangerous category. The current method that EOD teams use involves teleoperation of the robot to get from the base of operations to the object of interest. In the process of teleoperating the robot the human operator is exposed and vulnerable to external threats in the area such as ambushes and enemy snipers. As technologies mature to provide humans with better tools there are shortcomings discovered, such as the vulnerability due to teleoperation, that opens up avenues for improvement in the development of these tools. For robots one approach to reducing the amount of work humans are required to perform is to give the robots more intelligence via autonomous behaviors using additional sensors, specialized actuators and more advanced software to automate the largest amount of routine tasks as possible. In time more complex tasks become considered routine and are able to be automated as well.

When adding autonomy to robots nearly all of the tasks can be summarized by the following questions:
\begin{itemize}
\item Where am I?
\item What's around me?
\item Where do I want to go?
\item How do I get there?
\end{itemize}

The initial attempt at adding autonomy to the EOD robots resulted in somewhat erratic driving behavior, especially near obstacles as the robot trajectory would not be smooth when it changed speeds and attempted to make small corrections to its original path to move around the obstacle. In this thesis we will look at smoothing out the trajectories taken by the robot by looking for improvements in the state estimation (Where am I?) and controls (How do I get there?) algorithms. This work ignores actual obstacle detection (What's around me?) and will be using a simple planning algorithm (Where do I want to go?) to simulate obstacles in the robot path which will force the robot to change direction and speed multiple times.

Other tasks for small UGVs include sending them into buildings that are dangerous due to structural damage or unknown, possibly hostile elements inside so that the robots can map the interior and additionally to provide human operators with images to assess the danger prior to any humans entering the buildings \cite{CongressUGV06}. After the attacks on the World Trade Center on September 11, 2001 several small UGV systems were used to look for survivors in the rubble and to help assess structural damage to nearby buildings \cite{Everett02}.

For all of the tasks that robots are assigned to speed, efficiency and precision are important characteristics that make the difference between accomplishing or failing at that task. Improved autonomous navigation would improve all of those characteristics in small UGV systems. An example of an advanced behavior that becomes possible with better navigation is the ability to have a robot retrotraverse, or find its way back home. The benefits of improved navigation include less time waiting to get a robot close to an IED or to clear a building which would result in less time for humans in a hostile or dangerous environment. In search and rescue situations this would lead to less time for searching and more time for rescuing.

\section{Thesis Outline \& Contributions}
\label{sec:outline}
The problem of autonomous navigation is not isolated to any one technical area but is instead a combination of sensor integration to allow the robot position and obstacles to be observable, noise filtering, state estimation and control algorithms. As much as possible this research has attempted to isolate the effects of each of those areas to enable a quantitative analysis to determine which parts of the system contribute the largest effect on overall robot behavior.

Chapter \ref{ch:background} gives some background on the types of robots used, the sensors on the robots and the operator control unit (OCU). Chapter \ref{ch:estimation} presents the state estimation algorithm that is used on the robots. Chapter \ref{ch:controls} discusses the original control algorithm and a new control algorithm applied as part of this research. Chapter \ref{ch:results} gives the results of experiments run with the robots and looks at the main contributing factors for smooth autonomous navigation. Chapter \ref{ch:futurework} suggests future avenues of research to pursue. The conclusion is found in Chapter \ref{ch:conclusion}.

The main contribution of this research is that the small UGVs investigated here have greater autonomy which will allow for less human oversight of basic functionality. This is especially important for the dirty, dangerous and boring tasks where robots are most useful because it allows humans to focus their concentration, time and effort on being clean, safe and efficient.